\documentclass{article}
\usepackage[preprint]{neurips_2024}
\usepackage[utf8]{inputenc} % allow utf-8 input
\usepackage[T1]{fontenc}    % use 8-bit T1 fonts
\usepackage{hyperref}       % hyperlinks
\usepackage{url}            % simple URL typesetting
\usepackage{booktabs}       % professional-quality tables
\usepackage{amsfonts}       % blackboard math symbols
\usepackage{nicefrac}       % compact symbols for 1/2, etc.
\usepackage{microtype}      % microtypography
\usepackage{xcolor}         % colors
\usepackage{natbib}
\bibliographystyle{apalike}


\title{Cross-Region Information Flow During Visual Stimuli}
\author{%
    Adam Pitner\\
    Department of Psychology\\
    New York University\\
    \texttt{ap6231@nyu.edu}\\
    \And Judy Yang\\
    Center for Data Science\\
    New York University\\
    \texttt{hy1331@nyu.edu}}


\begin{document}
\maketitle
\begin{abstract}
    
\end{abstract}
\section{Introduction}
\subsection{Traditional View of Visual Processing Hierarchy}
The mammalian visual system has long been conceptualized as a hierarchical feedforward network. In the primate model, visual information flows sequentially from the retina through the lateral geniculate nucleus (LGN) of the thalamus to primary visual cortex (V1), then progressively through higher-order visual areas (V2, V4, MT) with increasing receptive field complexity and processing specialization \citep{felleman1991distributed, mishkin1983object}. This hierarchical organization, supported by anatomical tract-tracing and electrophysiological studies, suggests that each processing stage adds computational sophistication: V1 detects edges and orientations, V2 processes contours and textures, while higher areas encode complex features like object identity and motion \citep{hubel1962receptive, livingstone1988segregation}.

In rodents, particularly mice, the visual cortex exhibits a similar organizational principle with primary visual cortex (VISp, analogous to primate V1) receiving direct input from the dorsal lateral geniculate nucleus (LGd) and projecting to multiple higher visual areas (HVAs) including lateral (VISl), anterolateral (VISal), anteromedial (VISam), and posteromedial (VISpm) areas \citep{wang2007area, marshel2011functional}. Classical models predict sequential activation: LGd responds first (~30-40ms), followed by VISp (~40-60ms), then HVAs with progressively longer latencies (~60-100ms) as information ascends the hierarchy \citep{gao2010parallel}.

However, accumulating evidence challenges this strictly serial model. Anatomical studies reveal extensive parallel pathways and reciprocal connections between visual areas \citep{d2022hierarchical, siegle2021survey}. The lateral posterior nucleus (LP) of the thalamus, traditionally considered "higher-order," shows bidirectional connectivity with both VISp and multiple HVAs, potentially enabling parallel information streams that bypass the classical cortical hierarchy \citep{roth2016thalamic, bennett2019higher}. Furthermore, recent large-scale electrophysiology studies have documented near-simultaneous activation of multiple visual areas, suggesting that parallel processing may be more prevalent than previously appreciated \citep{siegle2021survey}.

\subsection{The Allen Brain Observatory Neuropixels Dataset}
The advent of high-density silicon probes has revolutionized our ability to record from large populations of neurons simultaneously across multiple brain regions. The Allen Brain Observatory Visual Coding - Neuropixels dataset \citep{AllenInstituteNeuropixels2019} represents a landmark resource, providing standardized recordings from over 40,000 neurons across 14+ brain regions in awake, head-fixed mice viewing diverse visual stimuli \citep{siegle2021survey, de2020large}. Each experimental session captures activity from primary and higher-order visual cortical areas, subcortical structures including thalamus, and even hippocampal regions, all recorded simultaneously with the same temporal reference.

This dataset offers unprecedented opportunities to examine visual information flow across brain networks. Unlike traditional studies that record from single regions in isolation, the multi-region recording capability enables direct comparison of response timing, trial-by-trial correlations, and directional information transfer between areas \citep{jun2017fully}. The standardized stimulus presentations—including drifting gratings, static gratings, natural images, and natural movies—allow systematic characterization of how different visual features propagate through cortical and subcortical networks.

\subsection{Present Study Objectives}
Despite the availability of this rich dataset, fundamental questions about visual information flow in the mouse brain remain incompletely resolved. Specifically: (1) Do visual areas activate in strict hierarchical sequence, or is there evidence for parallel input pathways? (2) How do cortical areas interact with thalamic nuclei—does the LP nucleus receive feedforward input or cortical feedback? (3) Which analytical approaches (cross-correlation, Granger causality, spike-triggered averages) most effectively reveal directional information flow between brain regions?

Here we leverage the Allen Neuropixels dataset to systematically characterize cross-region information flow during visual stimulation. We employ complementary analytical methods to measure response latencies (identifying temporal sequence), pairwise correlations (revealing functional coupling), and directed causal relationships (testing information transfer). Our goal is to construct a data-driven model of how visual information cascades through mouse cortical and subcortical networks, providing insights into the organizational principles governing sensory processing in the mammalian brain.
\bibliography{citation}
\end{document}
